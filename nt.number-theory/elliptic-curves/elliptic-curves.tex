\documentclass[12pt]{article}
    \usepackage{indentfirst}
    \usepackage{amsmath}
    \usepackage{amssymb}
    \usepackage{amsthm}
    \usepackage[utf8]{inputenc}
    \usepackage{geometry}
    \usepackage{diagbox}
	\usepackage{enumerate}
    \usepackage{siunitx}
    \usepackage{graphicx}
    \usepackage{multirow}
    \usepackage{xcolor}
    \usepackage{tikz-cd}
    \usepackage{mathrsfs}
    \usepackage{bm}
    \usepackage[colorlinks = true,
            linkcolor = blue,
            urlcolor  = blue,
            citecolor = blue,]{hyperref}
    \usepackage{cleveref}
    \title{Notes on Elliptic Curves}
    \author{Yourong Zang}
	\newtheorem{theorem}{Theorem}[subsection]
	\newtheorem{lemma}{Lemma}[subsection]
	\newtheorem{corollary}{Corollary}[subsection]
	\theoremstyle{remark}
	\newtheorem*{remark}{Remark}
	\newcommand*{\lemmaautorefname}{Lemma}
	\newcommand*{\corollaryautorefname}{Corollary}
    \newcommand{\res}[2]{\underset{#1}{\,\operatorname{Res}\,}#2}
    \newcommand{\ord}[0]{\operatorname{ord}}
    \newcommand{\ind}[0]{\operatorname{ind}}
    \newcommand{\w}[0]{\omega}
    \newcommand{\ve}[0]{\varepsilon}
    \newcommand{\s}[0]{\sigma}
    \newcommand{\D}[0]{\Delta}
    \newcommand{\Z}[0]{\mathbb{Z}}
    \newcommand{\R}[0]{\mathbb{R}}
    \newcommand{\F}[0]{\mathbb{F}}
    \newcommand{\N}[0]{\mathbb{N}}
    \newcommand{\Q}[0]{\mathbb{Q}}
    \newcommand{\C}[0]{\mathbb{C}}
    \newcommand{\A}[0]{\mathbb{A}}
    \newcommand{\Lam}[0]{\Lambda}
    \newcommand{\coker}[0]{\operatorname{coker}}
    \newcommand{\kp}[0]{\kappa}
    \newcommand{\doubp}[1]{\left(\left(#1\right)\right)}
    \newcommand{\lbd}[0]{\lambda}
    \renewcommand{\Re}[0]{\operatorname{Re}}
    \renewcommand{\Im}[0]{\operatorname{Im}}
    \newcommand{\nS}[0]{\mathcal{S}}
	\newcommand{\M}[0]{\mathcal{M}}
    \newcommand{\To}[0]{\mathbb{C}/\Lambda}
    \newcommand{\Too}[0]{\mathbb{C}/\Lambda'}
    \newcommand{\mtx}[4]{\begin{bmatrix}#1 & #2\\ #3 & #4\end{bmatrix}}
    \newcommand{\vp}[0]{\varphi}
    \newcommand{\norm}[1]{\left\lVert#1\right\rVert}
    \newcommand{\proj}[0]{\operatorname{proj}}
    \newcommand{\lcm}[0]{\operatorname{lcm}}
    \newcommand{\leg}[2]{\left(\frac{#1}{#2}\right)}
    \newcommand{\sgn}[0]{\operatorname{sgn}}
    \newcommand{\mult}[0]{\operatorname{mult}}
    \newcommand{\ft}[0]{\mathscr{F}}
    \newcommand{\rad}[0]{\operatorname{rad}}
    \newcommand{\Spec}[0]{\operatorname{Spec}}
	\newcommand{\Proj}[0]{\operatorname{Proj}}
    \newcommand{\MaxSpec}[0]{\operatorname{MaxSpec}}
    \newcommand{\Gal}[0]{\operatorname{Gal}}
    \newcommand{\im}[0]{\operatorname{im}}
    \newcommand{\Hom}[0]{\operatorname{Hom}}
    \newcommand{\height}[0]{\operatorname{height}}
    \newcommand{\id}[0]{\operatorname{id}}
    %\newcommand{\ord}[0]{\operatorname{ord}}
    \newcommand{\comment}[1]{}
    \newcommand{\Top}[0]{\mathsf{Top}}
	\newcommand{\Sch}[0]{\mathsf{Sch}}
    \newcommand{\op}[0]{\mathsf{op}}
    \newenvironment{psmallmatrix}
	{\left(\begin{smallmatrix}}
	{\end{smallmatrix}\right)}
	\newenvironment{bsmallmatrix}
	{\left[\begin{smallmatrix}}
	{\end{smallmatrix}\right]}
	\setcounter{section}{-1}
\begin{document}
    \maketitle
    \tableofcontents
    \newpage
    \section{Introduction}\label{sec-intro}
        The main reference of this note is Silverman's \textit{Arithmetic in Elliptic Curves}, and sometimes with examples from \textit{Rational Points on Elliptic Curves} by Silverman \& Tate.

	\section{Basic Constructions}\label{sec-basic}
        \subsection{Geometry}\label{ssec-geometry-basic}
        In order to study the rational solutions of a certain type of equations, it is important to construct our geometric objects in a Galois-theoretic flavor. Say $k$ is a perfect field\footnote{Note: to characterize the rational points with Galois theoretic-languages, we would only need a separable and possibly infinite algebraic closure (then it's algebraic and normal) so that the fixed field of the Galois group is exactly the ground field $k$; but I don't know how Silverman would use this condition, so I will leave it here.} and $K$ an algebraic closure of $k$, we denote by $\Gal(K/k)$ the Galois group of $K/k$. We can define a natural group action of $\Gal(K/k)$ on the affine space $\A^n$ over $K$ (we will always reserve the notation with subscripts for the affine schemes) by $(x_1,\dots, x_n)^\s =(x_1^\s,\dots, x_n^\s)$ for any $\s$ in the Galois group.

        Introduce the following notations. The \textit{set of $k$-rational points} $\A^n(k)$ is the invariant set of $\Gal(K/k)$. It is also clear that for any point $P$ and $f\in k[x_1,\dots, x_n]=k[X]$ we have $f(P^\s)=f(P)^\s$ for all $\s\in\Gal(K/k)$ since by constructions all $\s$ restricts to the identity on $k$. Denoted by $V/k$, an algebraic set $V$ \textit{is defined over $k$} if its ideal $I(V)$ is generated by polynomials with coefficients in $k$. The set of $k$-rational points of $V$ is then written as $V(k)$. It is therefore reasonable to define the ideal $I(V/k)=I(V)\cap k[X]$ (for any $V$), from which one can immediately see that $V/k$ iff $I(V)=\left(I(V/k)\right)$.

        If $V/k$, then as $I(V)$ is generated by polynomials in $k[X]$, elements of $\Gal(K/k)$ map points in $V$ to $V$. Therefore, we have $V(K)=V^{\Gal(K/k)}$. We can define the affine coordinate ring of a variety $V/k$ to be
        \[k[V]=\frac{k[X]}{I(V/k)}\]
        which is an integral domain if we require varieties to be irreducible. Similarly we define the function field $k(V)$ to be the field of fraction of $k[V]$.

        The Galois group clearly acts on $K[V]$: for any $\sigma\in\Gal(K/k)$ and $f\in K[V]$, $f^{\s}$ is defined to be the polynomial function with $\Gal(K/k)$ acting on $f$'s coefficients. Therefore, for any $f\in K[V]$ and $P\in V$, we have
        \[\left(f(P)\right)^\s=f^\s\left(P^\s\right)\]
    
        The constructions for projective varieties are similar. Since the Galois group consists of field automorphisms, it can act on homogeneous coordinates. For any point $P=[x_0,\dots,x_n]$ in the $n$-projective space, We define an extra object called the \textit{minimal field of definition for $P$ over $K$}, defined by
        \[k(P)=k(x_0/x_i,\dots, x_n/x_i)\]
        for any $x_i\neq 0$.
        \begin{lemma}
            Let $H$ be the subgroup of $\Gal(K/k)$ that fixes $P$. Then $k(P)=K^H$.
        \end{lemma}
        \begin{proof}
            \textcolor{red}{\textbf{Insert proof here!}}
        \end{proof}
        
\end{document}
